\documentclass[12pt]{article}
\usepackage[margin = 2.4cm]{geometry} % For margins of 3cm
\usepackage{graphicx}
\usepackage{float} % For H float position
\usepackage{gensymb} % For some symbols
\usepackage{amsfonts, amssymb, amsmath} % All three for maths symbols
\usepackage[export]{adjustbox} % For figure frames
\setlength{\parskip}{6pt} % To make nice looking paragraph spacing
\usepackage[export]{adjustbox} % For figure frames
\usepackage{rotating}
\usepackage[section]{placeins}
\usepackage{setspace} % For double spacing
\usepackage{pdfpages} % Allows including PDFs
\usepackage[sort&compress]{natbib} % bibliographies
\doublespacing

\title{RHEP background}
\date{}

\begin{document}
	\maketitle
\paragraph{Miller's syndrome and \textit{DHODH} mutation}
~\\ Miller's syndrome is rare autosomal recessive disoder characterised by craniofacial and postaxial limb deformities. Patients with Miller's syndrome present with severe micrognathia, cleft lip and/or palate, hypoplasia or aplasia of the posterior elements of the limbs. Normal intelligence is typical and internal malformations are rare.

Compound heterozygous missense mutations in the protein coding regeions of the \textit{DHODH} were identified as the cause of Miller's syndrome using whole-exome sequencing. Additionally biallelic mutations in \textit{DHODH} were identified in a further four unrelated families with typical clinical features of Miller syndrome.  \textit{DHODH} gene contains nine exons that encode a 43-kDa protein dehydroorotate dehydrogenase (DHODH). DHODH is a key enzyme in the \textit{de novo} pyrimdine synthesis and  localises in the inner mitochondrial membrane. DHDOH catalyses the oxidation of DHO to orotate and links it to the mitochondrial respiratory chain (MRC). The lack of homozygous mutations and the paucity of nonsense or frameshift alleles are unusual in a rare autosomal recessive disorder. 

A total of 14 different mutations in the coding regions of \textit{DHODH} have been reported, including 2 nonsense mutations. Evidence supporting the loss of the enzymatic activity of DHODH as the cause of Miller's syndrome was provided by the teratogenic effect of specific DHODH inhibitors in mouse embryos. Embryos exposed to the inhibitor showed a highly penetrant limb and craniofacial malformations. Rainger et al confirmed loss of enzymatic function by using complementation assays, demonstrating reduced pyrimidine synthesis and \textit{in vitro} enzymatic assays showing reduced enzyme activity in 11 disease-associated missense mutations. Three missense mutations were further evaluted by Fang et al. The three mutant proteins retained the proper mitochondrial localisation, however the proteins showed reduced protein stability. The allele c.403C>T;p.R135C has been reported in 3 families, and one indiviual. The mutation in R135C lies in the ubiqionone-binding site showed impariment of the substrate-induced enzymatic activity. No individual has been identified in which both alleles show severe loss-of-function.  

It is interesting that the process of \textit{de novo} pyrimidine synthesis leads to such tissue specific phenotypes. However, analysis of the mouse ortholog \textit{Dhodh} expression in mouse embryo showed spatio-temporal specific activity in pharyngeal arches and limb-buds, consistent with the regions affected in MS indicating DHODH mutations, and subsequent LOF may specifically affect the embryo.

\paragraph{Role of p53 during development}
~\\ Leflunomide (Arava), and its metabolite teriflunomide have been used, in various studies, to inhibit DHODH. The use of DHODH inhibitors leads to the impairment of pyrimidine synthesis, and causes mitochondrial dysfunction causing the depletetion of pyrimdine neucleotide pool, and generation of reactive oxygen species, respectivetly. Subsequently, these effects contribute to an increase in p53 levels.

p53 is considered a master regulator of proliferation, apoptosis, and has been linked to cell differentiation in a vareity of cell types. The effect of p53 has been explored in Treacher-Collins syndrome, another craniofacial disoder. Tcof1 mutations led to an increase in p53 levels in the neuroepithelium in mouse embryos. Knock-out of Trp53 in the TCS mouse model amerliorated the craniofacial abnormalitles. Additionally, p53 has shown to be involved in neural crest development. Stabilization of p53 protein resulted in fewer migrating CNC cells and in craniofacial defects in chick and mouse embryos, showing that p53 coordinates CNC cell growth by affecting cell cycle gene expression and proliferation at discrete developmental stages. These observations indicate that DHODH LOF may cause p53 up-regulation during embryonic development. 


\paragraph{Craniofacial and limb development}
 ~\\The primary phenotype associated with MS diagnosis is defective craniofacial and limb skeleton development. Skeletal development begins when loose networks of mesenchymal cells coalesce and condense, prefiguring mature cartilage and bone. Different mesenchymal populations give rise to anatomically distinct groups of bones. Neural crest-derived mesenchyme forms bones in the face, jaw and rostral calvarium whereas mesoderm-derived mesenchyme forms bones in the caudal calvarium, vertebral column, rib cage, and limbs. Osteoblasts differentiation is triggered by a variety of intra- and extracellular osteogenic signals, and  requires regulation by several transcription factors. Among these, the transcription factors Runx2 and Osterix/SP7 have critical roles in osteogenesis. 
~\\Previously, MC3T3-E1 cells had been used to show that depletion of DHODH activity decreased osteogenic gene expression. Expression and protein levels of p53 showed an increase in this study, however its relation to osteoblast differentiation was not explored. MC3T3-E1 cells contains Osx/SP7. Conditional inactivation of Osx/SP7 in cranial neural crest cells of mice has shown defects in craniofacial bone development. Additionally, a human patient with a homozygous mutation in Osx/SP7 displayed craniofacial and limb bone deformities suggesting roles of Osx/SP7 in bone differentiation and patterning during embryonic development. p53 is able to exert a repressive effect on Osx/SP7 leading to disruption in osteoblast differentiation. p53 

\textit{Osx} mRNA levels show significant up-regulation between E11.5 and E13.5 during mouse embryonic development, subsequent to the increased \textit{Dhodh} expression observed at E10.5. Therefore one of the mechanisms by which DHODH LOF may cause the MS phenotype is by inhibiting  Osx/SP7  due to p53 over-expression during craniofacial and limb development.







%It is worth noting that studies of teratogenicity of leflunomide in mice showed a reduction in fetal viability and an increased incidence of multiple external, skeletal, and visceral malformations [153], common features of a disorder referred as to as Miller syndrome, a rare genetic abnormality described by craniofacial malformations together with abnormalities of the arms, hands and/or feet [154–157]. Those findings serendipitously allowed researchers to link biallelic mutations in the gene that encodes DHODH to Miller syndrome, also known as Genee-Wiedemann syndrome, Wildervanck-Smith syndrome, or post axial acrofacial dystosis (POADS). The data compiled by Rainger and collaborators suggest that DHODH activity, and consequently pyrimidine biosynthesis, is essential throughout human and mammalian facial and limb development, and deficiency could present as the clinical characteristics of Miller syndrome, or even embryonic lethality 
%~\\ DHODH inhibition has been used in various auto-immune therapy and cancer therapy studies.  The inhibition also contributes to increased levels of p53. p53 is a global transcription factor with roles in apopotsis, as a transcription regulator etc etc. The effects of p53 in craniofacial development has mainly been explored in the Treacher-Collins syndrome (TCS) 

%The effect of p53 has been explored in Treacher-Collins syndrome (TCS), another craniofacial disorder. Trp53 knock-out in TCS mouse models ameliorated the craniofacial anomalies showing that p53 plays a role TCS pathology. These observations indicate that p53 over-expression in embryo leads to craniofacial abnormalities. \textit{DHODH} mutations may cause p53 up-regulation either via pyrimidine deficiency, mitochondrial dysfunction or in combination during embryonic development. 
 
 %Miller syndrome was the first Mendelian disorder whose molecular basis was identified via whole-exome sequencing and shown to correlate with mutations in dihydroorotate dehydrogenase (DHODH). Recently additional biallelic mutations in DHODH were identified in a further four unrelated families with typical clinical features of Miller syndrome . To date, a total of 14 distinct mutations in the coding region of DHODH, including two nonsense mutations, have now been identified. All affected individuals have compound heterozygous mutations in the DHODH coding region. The lack of homozygous mutations and the paucity of nonsense or frameshift alleles are unusual in a rare autosomal recessive disorder. 
 
 %A total of 14 different mutations in the coding region of DHODH, including 2 nonsense mutations, have been reported in Miller's syndrome. The allele c.403C>T;p.R135C has been reported in 3 families, and one indiviual. 
 
 %Loss of the enzymatic activity of DHODH as the cause of Miller's syndrome is provided by the teratogenic effect of specific DHODH inhibitors in mouse embryos. Additionally, enzymatic assays of Miller's syndrome based mutations of have led to protein instability and loss of DHODH activity.
 %Embryos exposed to leflunomide show a highly penetrat limb and craniofacial malformations. Additionally, enzymatic assays 
 
 %Additionally, Rainger et al showed used In vitro DHODH enzymatic assays to quantitate the loss of enzymatic function of all 11 missense mutations found in Miller's syndrome patients. Each of the Miller's syndrome associated alleles tested displated significantly reduced actvity compared with the wild-type human protein.
 
 %assays to show thata te DHODH missense mutations showed reduced growth compared to wild-type DHODH. 
 
 %identified compound heteozygous missense mutations in the protein coding regions of the \textit{DHODH} gene (illustrated in Figure 1). .
 


	\end{document}